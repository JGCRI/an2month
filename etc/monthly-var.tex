\documentclass[11pt]{article}
\usepackage{amsmath}

\begin{document}
\title{Variability in downscaling annual climate totals to monthly}
\author{Robert Link}
\date{2018-12-27}
\maketitle

\begin{abstract}
In the current version of \texttt{an2month}, our monthly fractions are
fixed, based on the average values we extract from the ESM data we are
analyzing.  This is not ideal, especially for the analysis of drought
risk that we are planning to do in Paper 1.  Variation in the
distribution of rainfall and temperature throughout the year surely
play a role in defining the tails of the climate distribution we are
trying to generate, and it seems likely that this variability is no
less important to those tails than the year-to-year variation in
annual averages and totals.

The hard thing in generating variable monthly totals is that the
monthly values are not independent of one another; they \emph{must}
add up to the annual total.  Generating constrained random values
directly is difficult, so instead we will derive a way to estimate the
parameters of an \emph{unconstrained} multivariate distribution that
can be transformed into a distribution that respects the constraints.
\end{abstract}

Let $x_i$ be the observed monthly fractions of the annual total for a
single year. There will be $N$ such values,
$\left\lbrace x_1, x_2, \ldots, x_N\right\rbrace$.
For monthly fractions $N=12$, but the process is the same no matter
the value of $N$, so we'll leave it unspecified.

Our constraints are that $\sum_{i=1}^{N} x_i = 1$, and for all $i$,
$0 \le x_i \le 1$.  We can rewrite $x_i$ in terms of
\emph{unconstrained} variables $z_i$ using the transformation
\begin{equation}
  \label{eq:softmax}
  x_i = \frac{e^{z_i}}{\sum_{j=1}^{N} e^{z_j}}.
\end{equation}
This function is commonly called the \emph{softmax} function.  We can
easily show that for any combination of $z_i \in (-\infty, \infty)$,
the resulting $x_i$ are between zero and 1, and they sum up to 1,
satisfying our constraints.

The transformation in equation~(\ref{eq:softmax}) is not unique.  We
can show that if we add the same constant $\kappa$ to all of the
$z_i$, then $e^\kappa$ can be factored out of both the numerator and
the denominator, leaving the $x_i$ unchanged.  This is a direct
consequence of the fact that, although we have $N$ values, the
constraint that they must sum to 1 means that there are only $N-1$
\emph{independent} values.  Once you have chosen those, the last value
is fixed at 1 minus the sum of the others.

To deal with this ambiguity, define
\begin{equation}
  z_N \equiv 0.
\end{equation}
This definition allows us to invert equation~(\ref{eq:softmax}) to get
the $z_i$ from the observed $x_i$.  We do this by working in terms of
the log-ratio of $x_i$ and $x_N$.
\begin{equation}
  \ln\left(\frac{x_i}{x_N}\right) = z_i - z_N = z_i,
\end{equation}
where the last step follows because we have defined $z_N$ to be zero.

We are now in a position to get an empirical distribution for the
$z_i$ from our observed $x$ values.  Let $\vec{x}^{(k)}$ be the set of
$x_i$ values for the $k$th year of observations, $k \in [1,M]$.  From
these, we can compute the corresponding set of $\vec{z}^{(k)}$.  For
each month $i$, the $z_i^{(k)}$ provide an empirical distribution.
Therefore, to generate a single year's downscaling factors, we have
only to draw a random set of $z_i$, \emph{independently} from those
distributions, and compute the corresponding $x_i$ using
equation~\ref{eq:softmax}.

We still have to find a way to represent these distributions.  There
will be one for every month, in every grid cell, so it would be best
to keep the description as compact as possible.  My hope is that the
$z_i$ will be approximately normally distributed, so we could just
compute the mean and variance and store them for each combination of
month and grid cell.  If a normal distribution is \emph{not} a good
approximation for the distribution of the $z_i$, then we can add a
couple of parameters to turn the distribution into a generalized
normal distribution (GND), which should be close enough to the real
distribution for what we are trying to do here.  Unfortunately, it
looks like there isn't a way to estimate the parameters for a GND in
closed form from a set of samples, so if we end up having to go that
route, we will need to put together a numerical parameter estimation
subroutine.

Adding this variability to the monthly downscaling will complicate the
\texttt{an2month} package somewhat, but the payoff will be that we
will have true monthly variability in our generated data, rather than
a fixed seasonal variability that merely scales up and down with
changes in the annual totals.

\end{document}
